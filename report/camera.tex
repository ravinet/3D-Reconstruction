\section{Camera Calibration}
\label{s:camera}

To calibrate our camera and obtain its intrinsic parameters, we use Zhang’s method. 
Zhang’s method is a technique that uses several images to derive a camera’s focal length, aspect ratio, and principal points. Whilethere are many other approaches to performing this essential task, Zhang’s method is more flexible and robust. The two traditional approaches are often categorized as either photogrammetric or self-calibration. Photogrammetric methods require using a 3D object whose 3D coordinates are precisely known. Taking multiple images of this 3D object lets one infer the intrinsic parameters of the camera as they can be derived from the difference between the actual coordinates of the object and what is seen across the images. However, the apparatus necessary to perform this type of calibration is expensive. Self-calibration, while less costly, is not reliable as there are often not enough known points to estimate all the necessary parameters. Self-calibration requires moving the camerain a static setup and performing a feature points matching across the taken images. Self-calibration also uses constraints pertaining to the rigidity of the object considered.
Zhang’s method is a cross between photogrammetric and self-calibration techniques. Zhang’s method requires one to construct a pattern on a paper and attach it to a planar surface. Several images of the pattern, from different angles, are then taken. As long as the camera or the pattern is stationary, the movement is not restricted. Using an understanding of the geometry of the designed pattern, constraints on the intrinsic parameters arise from each view. Using all of these constraints, a set of intrinsic parameters which satisfy all the considered views can be inferred. Of course, using more views will yield intrinsic parameters closer to their actual values. Thus, Zhang’s method incorporates understanding 3D coordinates of the pattern  (photogrammetric) and using multiple views to set constraints on the intrinsic parameters (self-calibration), but it only considers one plane and a user-designed pattern. In their paper, Zhang et. al. show that their calibration technique is both flexible, robust, and accurate. 
A camera’s intrinsic matrix is defined as: 
A = row 1( alpha, gamma, u0), row 2(0, beta, v0)  row3( 0, 0, 1) where (u0, v0) is the principal point, alpha and beta are image scale factors, and gamma is a parameter which describes the skew in the image axes. We use the pinhole camera model which dictates that a 3D point M is related to its image projection, m, by: sm = A[R t]M where R is the rotation matrix and t is the translation vector of the considered image. 
For each view used, we estimate a homography matrix, H = [h1 h2 h3] = lambda*A[r1 r2 t], where r1 and r2 are the components of the rotation matrix, R, and lambda is a scaling factor.  Since r1 and r2 are orthonormal, we can derive the following two constraints:
EQUATIONS 3 AND 4 FROM SECTION 2.3 OF ZHANG PAPER
The approximation of the camera’s intrinsic parameters is improved using a maximum-likelihood estimation. Zhang’s method also extracts the camera’s extrinsic parameters for the given input images, but we do not make use of this data as we are reconstructing an object separate from the considered planar pattern. 
