\section{Camera Calibration}
\label{s:camera}

To calibrate our camera and obtain its intrinsic parameters, we use Zhang’s method. Zhang’s method is a technique that uses several images to derive a camera’s focal length, aspect ratio, and principal points. While there are many other approaches to performing this essential task, Zhang’s method is more flexible and robust. The two traditional approaches are often categorized as photogrammetric or self-calibration. Photogrammetric methods require using a 3D object whose 3D coordinates are precisely known. Taking multiple images of this 3D object lets one infer the intrinsic parameters of the camera as they can be derived from the difference between the actual coordinates of the object and what is seen across the images. However, the apparatus necessary to perform this type of calibration is expensive. Self-calibration, while less costly, is not reliable as there are often not enough known points to estimate all the necessary parameters. Self-calibration requires moving the camera in a static setup and performing a feature points matching across the taken images. Self-calibration also uses constraints pertaining to the rigidity of the object considered.
Zhang’s method is a cross between these two classes of camera calibration. Zhang’s method requires one to construct a pattern on a paper and let the paper rest on a planar surface. Several images, from different views, are then taken of the pattern. As long as the camera or the pattern is stationary, the movement is not restricted. Using an understanding of the geometry of the designed pattern, constraints on the intrinsic parameters arise from each view. Using all of these constraints, a set of intrinsic parameters which satisfy all the considered views can be inferred. Of course, using more views will yield intrinsic parameters closer to their actual values. Thus, Zhang’s method incorporates understanding 3D coordinates of the pattern  (photogrammetric) and using multiple views to set constraints on the intrinsic parameters (self-calibration), but it only considers one plane and a user-designed pattern. In their paper, Zhang et. al. show that their calibration technique is both flexible, robust, and accurate. 
The first step of Zhang’s method is to obtain images of different views of the constructed planar pattern. For each view, a homography matrix is calculated analytically. It is improved using a maximum-likelihood estimation.  Using these homography matrices, we can solve for 6 intermediate parameters that contain the five necessary intrinsic parameters of the camera. After extracting the intrinsic parameters, their accuracy is improved using maximum-likelihood estimation.

