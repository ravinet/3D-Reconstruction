\label{s:conclusion}
This project describes a flexible and practical 3D-reconstruction technique. Rather than having to perform a camera calibration for each input image used in the reconstruction, as many previous techniques have required, we only perform the calibration once using Zhang’s method. Thus, this approach scales well with the number of input images used, as long as they are all taken with the same camera. We also combine several common feature point extraction methods to enhance the detail in the final 3D reconstruction. Using both SIFT and Harris corner detection, we are able to get enough pairs of matching points to recover all eight unknowns in the fundamental matrix while incorporating all the necessary feature points to capture depth and structural details in the reconstructed object. The benefits of including Harris corner detection are highlighted in the results we provided. Both edges and corners are much more pronounced in the reconstructed images using corner detection than in those when we used SIFT alone. 

