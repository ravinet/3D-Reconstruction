\section{Geometry of 3D Reconstruction}
\label{s:fundamental}
In order to understand the purpose of the fundamental and essential matrix, we first describe epipolar geometry. Epipolar geometry captures the intrinsic projective geometry between two views of a scene, which only depends on the internal parameters of the camera and relative positions of the two views. The fundamental matrix contains this information, and from this we can derive the essential matrix, which gives us the relative position and orientation of the two camera views. Using the essential matrix, we can derive  3D points from 2D points in both views.

\subsection{Epipolar Geometry}
In our project, we take two uncalibrated pictures of the same object from different views. Epipolar geometry gives us a relationship between these images.
\begin{figure}[H]
\begin{center}
\includegraphics[width=0.9\linewidth]{figures/epipolar.pdf}
\end{center}
\caption{Epipolar geometry of two images from different views.}
\label{epipolar_pic}
\end{figure}

Figure~\ref{epipolar_pic} illustrates the relevant epipolar geometry. $I_1$ and $I_2$ are the image planes and $C$ and $C'$ represent the optical center of the camera from different views. $m$ and $m'$ represent $P$ projected onto $I_1$ and $I_2$ respectively. $e$ and $e'$ represent the epipoles, which are the intersections of the image planes with the line that connects the camera centers. $l_m$ and $l_m'$ are the epipolar lines, which is where the epipolar plane intersects the image plane. It is important to note that $m$ and $m'$ lie on $l_m$ and $l_m'$ respectively. As a result, we know that the matching point for $m$ is going to be on the line $l_m'$ in $I_2$ instead of anywhere in space of $I_2$. In order to capture this geometric restriction, we use the fundamental matrix, which we will describe in the next section.

\subsection{Fundamental Matrix}
The fundamental matrix, $F$, is a mathematical representation of the epipolar geometry described above. From Figure~\ref{epipolar_pic}, we can see that for a point $m$ in $I_1$, there is a corresponding epipolar line $l_m'$ in $I_2$ and the matching point $m'$ must lie on that line. We can think of this as a mapping from a point to a line, specifically a projective mapping from points to lines. This projection is represented by the fundamental matrix $F$. Algebraically, two matching points $m$ and $m'$ on two images must satisfy the following relation:
\begin{equation}
m^TFm = 0
\end{equation}

There are many methods to find the fundamental matrix for a specific pair of images algebraically and geometrically. Each method has its tradeoffs for time, complexity, and error. For the sake of space, we provide a high level overview of how we calculated the fundamental matrix as many of the details vary based on implementation. The fundamental matrix is a 3x3 matrix, so there are 9 parameters. However, after normalizing on one parameter, we only need to find 8 parameters. This dictates that we need at least 8 pairs of matched feature points to construct the fundamental matrix. To obtain points for corresponding features in the two images automatically we use SIFT and Harris corners as described in Section~\ref{s:sift}. We consistently generate many more than 8 matched feature points. We use RANSAC~\cite{ransac} to filter out outliers and calculate $F$ such that there are the largest number of inliers. To calculate the fundamental matrix, we use the least median squares method~\cite{lms_zhang} to minimize error. 

We now want to obtain the extrinsic parameters (orientation and translation) from the fundamental matrix, which are found in the essential matrix.

\subsection{Essential Matrix}
The essential matrix gives us the extrinsic parameters of the two views. In other words, we find the translation and rotation of the views relative to one another. However, we can only use the essential matrix if we know the internal parameters of the camera. In section~\ref{s:camera}, we describe a method to find the internal parameters of the camera. Using the internal paramaters of the camera, we have the following relation between the essential matrix and fundamental matrix:
\begin{equation}
E = K^TFK
\end{equation}
where $F$ is the fundamental matrix and $K$ is the internal parameters of the camera.

Theoretically, the essential matrix should have two equal non-zero eigenvalues and a zero eigenvalue. However, due to the noise in the data, this usually does not hold in practice. In order to rectify this, we use Singular Value Decomposition (SVD)~\cite{svd} to capture the diagonal matrix with the eigenvalues. We set the smallest eigenvalue to 0 and set the other two eigenvalues as the average between them. Finally, with this diagonal matrix, we construct a new essential matrix.

Now, we have all the tools to perform the 3D reconstruction.
