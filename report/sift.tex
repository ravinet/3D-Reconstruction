\section{SIFT and Harris Corners}
\label{s:sift}

Here, we describe two forms of automated feature detection. First, we discuss the scale-invariant feature transform (SIFT)~\cite{SIFT}, which is an algorithm that detects local features in an image. For our purposes, we use SIFT to gather matching points between the two images. However, many times SIFT is insufficient to capture important features specific to the images. In our case, we found it difficult for SIFT to capture the corners, which define the shape of the object. As a result, we also applied Harris corners~\cite{Corners} and included them as features. The difference in the reconstructions with and without Harris corners are show in Section~\ref{s:results}.

SIFT is an algorithm that detects local features in images. It is widely used because of its robustness to geometrical changes as well as its ability to successfully extract stable feature points. Since we just use SIFT as a black box package, we will only provide a high level overview. Here are the four main steps in the SIFT algorithm: extreme point detection, accurate localization of key point, assignment the main orientation of key point, and the creation of key point descriptor. 

To detect the extreme point, a Difference of Gaussian (DOG) pyramid of the image is built, and a pixel is compared with its 26 neighbors in 3x3 regions. If the point is a maximum or minimum in the 26 neighbors in the DOG scale space, then it is a feature point. In order to assign the orientation of the key point, samples around the key point are taken. A gradient histogram is created from the gradient orientations of the sample points. From the histogram, we assign the highest peak as the orientation of the key point. Finally, the key point descriptor is created by sampling points with a 16x16 region around the key point and creating 8 orientation bins from each of the 4x4 subsections. From this, we can create seed points to give us a feature vector. SIFT decides that two key points are matching by checking the Euclidean distance between two feature vectors as well as the nearest neighbor algorithm. More specific details can be found in the paper~\cite{SIFT}.

Like with SIFT, we just provide a high level overview of finding Harris corners. A corner is defined as a point in the image where the gray changes drastically or the junction of the contour boundary. The Harris algorithm attempts to determine whether a given pixel is a corner by using the rate of gray scale change. 

These matching points are important to calculate the 3D geometry of our object.