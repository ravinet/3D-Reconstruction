\section{SIFT and Harris Corners}
\label{s:sift}

Here, we describe two forms of automated feature detection. First, we discuss the scale-invariant feature transform (SIFT)~\cite{SIFT}, which is an algorithm that detects local features in an image. For our purposes, we use SIFT to gather matching points between the two images. However, many times SIFT is insufficient to capture important features specific to the images. In our case, we found it difficult for SIFT to capture the corners, which define the shape of the object. As a result, we also applied Harris corners~\cite{Corners} and included them as features. The difference in the reconstructions with and without Harris corners are show in Section~\ref{s:results}.

SIFT is an algorithm that detects local features in images. It is widely used because of its robustness to geometrical changes as well as its ability to successfully extract stable feature points. Since we just use SIFT as a black box package, we will only provide a high level overview. Here are the four main steps in the SIFT algorithm: extreme point detection, accurate localization of key point, assignment the main orientation of key point, and the creation of key point descriptor. 

To detect the extreme point, a Difference of Gaussian (DOG) pyramid of the image is built, and a pixel is compared with its 26 neighbors in 3x3 regions. If the point is a maximum or minimum in the 26 neighbors in the DOG scale space, then it is a feature point. In order to assign the orientation of the key point, samples around the key point are taken. A gradient histogram is created from the gradient orientations of the sample points. From the histogram, we assign the highest peak as the orientation of the key point. Finally, the key point descriptor is created by sampling points with a 16x16 region around the key point and creating 8 orientation bins from each of the 4x4 subsections. From this, we can create seed points to give us a feature vector. SIFT decides that two key points are matching by checking the Euclidean distance between two feature vectors as well as the nearest neighbor algorithm. More specific details can be found in the paper~\cite{SIFT}.

We also use Harris corner detection as a means to improve our 3D reconstruction. A corner is defined as a point in the image where the gray changes drastically or the junction of the contour boundary. 

The Harris corner detection technique is an enhancement of Moravec corner detection. For each pixel in a considered image, a Moravec corner detector creates a patch centered at that pixel and compares it to neighboring patches which have large overlap. Neighboring patches are created at 45 degree shifts (perpendicular, parallal, diagonals). For each comparison, the algorithm computes the sum of squared differences between the two considered patches with respect to the difference in gray scale intensity. The algorithm is able to deduce what kind of feature point the pixel represents simply from several squared difference sums, which represent different orientations of compared patches. If the considered pixel is on a edge, the compared patches will differ from the considered patch significantly as we move perpendicular to the edge, and will be similar to the considered patch as we move parallel to the edge. Similarly, a pixel representing a corner will yield patches which significantly differ from the considered patch. However, the considered patch will differ greatly with all considered patches, regardless of orientation. If the considered pixel is not on an edge or a corner, the squared differences obtained will be relatively small regardless of orientation. 

Harris corner detection improves on Moravec corner detection in several ways. Harris corner detection considers shifts for the neighboring patches in all directions rather than just 45 degree shifts. Additionally, Harris corner detection uses a circular Guassian window for comparisons which reduces noise. As a result, Harris corner detectors are able to distinguish between edges and corners more accurately than Moravec corner detectors.   

These matching points are important to calculate the 3D geometry of our object.
